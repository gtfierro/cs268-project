\documentclass[conference]{IEEEtran}
\usepackage{enumitem}
\usepackage{graphicx}
\usepackage{url}
\usepackage{xcolor}
\usepackage{listings}
\usepackage[labelfont=bf,format=plain,font=small]{subcaption}
\usepackage[labelfont=bf,format=plain,font=small]{caption}
\usepackage{array}
\newcolumntype{L}{>{\arraybackslash}m{3cm}}
\usepackage{minted}
\newminted{sql}{mathescape, numbersep=5pt, frame=lines, framesep=2mm, fontsize=\footnotesize}
\newminted{lua}{mathescape, numbersep=5pt, frame=lines, framesep=2mm, fontsize=\footnotesize}
\usepackage{MnSymbol}
\def\prebreak{\raisebox{0ex}[0ex][0ex]{\ensuremath{\rhookswarrow}}}
\def\postbreak{\raisebox{0ex}[0ex][0ex]{\ensuremath{\hookrightarrow\space}}}
\def\lstbreak{\prebreak\newline\postbreak}
\lstdefinelanguage{pseudocode}{sensitive=false,morecomment=[l]{//},morestring=[b]",
                               morekeywords={if,else,while,continue,true}}
\lstset{breaklines=true,breakindent=5pt,
        postbreak=\postbreak,prebreak=\prebreak}

\newcommand{\todo}[1]{\textcolor{red}{TODO: #1}\PackageWarning{TODO:}{#1!}}


\begin{document}

\title{CS268 Project}

\author{\IEEEauthorblockN{Gabriel Fierro}
\IEEEauthorblockA{gt.fierro@berkeley.edu}
\and
\IEEEauthorblockN{Zain Amro}
\IEEEauthorblockA{zamro@berkeley.edu}
}

\maketitle

\begin{abstract}
TODO
\end{abstract}

\section{Introduction}

It has been several years since RPL was introduced as a routing protocol for the kinds of links used in wireless sensor networks (WSNs).
While various commercial products have introduced some form of IPv6 over LoWPAN, these operate in stand-alone settings, such as automated meter reading networks, and there is essentially nothing in the literature regarding specifics or performance of their routing protocols.
Open implementations of RPL are available, but there is a pronounced lack of study of their empirical behavior, particularly in the common case of small (10s of nodes) shallow (one to three hop) networks.

The original vision of this project was to develop a new routing protocol --- \texttt{3hop} --- for this common but understated scenario and to augment a discussion of the design and implementation of this new protocol with a comparative study of RPL.


\if 0
- Motivation starts off by saying: look, for the very common case of indoor, low-power
  sensor networks, there is a need for routing.
- what do WSN nodes look like?
    - http://www.airccse.org/journal/ijcses/papers/1110ijcses06.pdf
    - low bandwidth, low power, low memory, short range (10-50m)
    - unreliable -- things crash, run out of power -> topology changes!
      - need to adapt to failures
    - self-configuring: nodes will need to discover the topology.
    - channel utilization:
      - want to make efficient use of bandwidth by reducing control traffic
        necessary for maintaining the mesh
      - has upshot of saving power too
      - this is also addressed at MAC layer, which we do not address
- This particular case of WSN topology formation is defined by the following characteristics:
    - SMALL: order 10s of nodes, over a small geographic area: a building or collection of rooms
    - mains power available at some locations throughout a space: wall plugs here and there
    - It is feasible to cover the monitored space in a mesh with hops no greater than 3
    - environment can change: furniture, people -- transient connectivity
- establish the goals and parameters of our proposed routing protocol
- Firstly, what does a routing protocol have to do?
    - mote must discover the prefix
      - simplifying assumptions: slaac, using 48-bit MAC and 16 bit node id
    - build upward routes: multipoint-to-point
      - list some ways to do this?
    - build downward routes
      - need for acknowledgements, actuations
    - point to point routes
      - rarer, but in small networks w/ actuators and sensors, this may become more common.
    - prepare for failure:
      - provide a means of repairing the network
    - useful parts: border router, providing physical translation in/out of
      the network. Handles prefix compression/expansion for 6lowpan, generally
      more powerful, powered. Can have large bandwidth backbone
\fi



\section{Context and Motivation}
\if 0
- Motivation starts off by saying: look, for the very common case of indoor, low-power
  sensor networks, there is a need for routing.
- what do WSN nodes look like?
    - http://www.airccse.org/journal/ijcses/papers/1110ijcses06.pdf
    - low bandwidth, low power, low memory, short range (10-50m)
    - unreliable -- things crash, run out of power -> topology changes!
      - need to adapt to failures
    - self-configuring: nodes will need to discover the topology.
    - channel utilization:
      - want to make efficient use of bandwidth by reducing control traffic
        necessary for maintaining the mesh
      - has upshot of saving power too
      - this is also addressed at MAC layer, which we do not address
- This particular case of WSN topology formation is defined by the following characteristics:
    - SMALL: order 10s of nodes, over a small geographic area: a building or collection of rooms
    - mains power available at some locations throughout a space: wall plugs here and there
    - It is feasible to cover the monitored space in a mesh with hops no greater than 3
    - environment can change: furniture, people -- transient connectivity
- establish the goals and parameters of our proposed routing protocol
- Firstly, what does a routing protocol have to do?
    - mote must discover the prefix
      - simplifying assumptions: slaac, using 48-bit MAC and 16 bit node id
    - build upward routes: multipoint-to-point
      - list some ways to do this?
    - build downward routes
      - need for acknowledgements, actuations
    - point to point routes
      - rarer, but in small networks w/ actuators and sensors, this may become more common.
    - prepare for failure:
      - provide a means of repairing the network
    - useful parts: border router, providing physical translation in/out of
      the network. Handles prefix compression/expansion for 6lowpan, generally
      more powerful, powered. Can have large bandwidth backbone
\fi

\section{RPL -- Gabe }

\if 0
In this section, we need to establish what RPL is, a little tiny bit of its history, and then
go into its design, how it addresses the requirements above, and how well we can predict
these might fit into our chosen scenario.

What is RPL:
  - distance-vector routing protocol designed for low-power and lossy networks, a class of
    network in which both the routers and the interconnects are constrained
  - aims to provide all three traffic patterns: up, down. point-to-point:
    - contrast to CTP which only does collection (up)
  - "RPL does not rely on any particular features of a specific link-layer technology" -RFC6550
  - means that it cannot make assumptions about those features, meaning it can lead to
    redundant design decisions
  - here we concentrate on the application of RPL to low-power, lossy, short-range wireless
    sensor networks
  - brings with it following opportunities and constraints
    - harness the broadcast medium
    - be aware of bandwidth usage, talking too much
    - specifically, "repair" mechanisms can cause a lot of traffic and end up bringing down
      other nodes because now they are overloaded

How do it work though:
- topology:
  - RPL instance = set of 1+ DODAGs sharing RPLinstance ID:
    - each operate independently, implements different objective function
  - DODAGs identify roots. uniquely identified by combination of RPL instance ID
  - rank: defines nodes individual position relative to other nodes w/ respect to the
    DODAG root. Increases farther away from root
  - DODAG root: usually the border router, configures parameters, determines when global
    repair happens
- control mesages
- how topology formation works

We can then start criticizing it:
- complexity:
  - multiple DODAGs, RPL instances, OFs
  - secure version of all headers -- not clear this could be achieved using app level
    or L2 security. To knowledge no one has implemented this
- underspecification:
  - how often to send DAO messages? relayed to the parent, constant overhead of messages
  - interoperability is a problem (contiki/tinyrpl happy together). Two implementations
    must have the same feature set, which is different than the minimal set of features
    required to have the system working:
    - storing mode vs non storing mode
- unnecessary features:
  - RPL instances -- the standard does not even talk about multiple
  - multiple objective functions: only 2 were ever implemented
  - dio/dis are very similar to ipv6 nd ra/rs messages
\fi

\subsection{IPv6 Neighbor Discovery -- Gabe }
\if 0
- RPL is not the only standard that would need to be revisited for wireless LLNs.
- IPv6 Neighbor Discovery is intended to provide following features:
    - router discovery: find routers on attached link
    - parameter discovery:
        - prefix discovery, link MTU, hop limit, etc
    - stateless address autoconfiguration
    - address resolution
    - neighbor unreachability detection (NUD)
    - duplicate address detection
- these are more features than are needed for WSNs because of restrictions:
    - NUD: no need to maintain connectivity to neighbors
        - use the application traffic to determine if link is still good
        - if you don't use the link, don't waste time and energy keeping it up:
            - places pressure on the channel because you can only broadcast!
        - IP is best-effort: use this leeway!
    - duplicate address detection:
        - a common way of configuring motes is SLAAC
        - one way of doing this is 64-bit prefix + 64-bit EUI-64 address
        - 48bit MAC, last 16 bits are a unique node identifier
        - upshot: can compress the address of a mote into 2 bytes (2^16-1 addresses)
        - guaranteed unique address: no need to check for dups
- constants: too chatty?
    - rfc6775 attempts to adapt for 6LowPANs:
    - "Minimize signaling by avoiding the use of multicast
      flooding and reducing the use of link-scope multicast messages"
    - original constants: cite them
    - rfc6775 new constants:
        - max RAs: 3
        - min delay btwn RA: 10s
        - max RA delay time 2 sec
        - RS interval: 10s
        - max RS : 3
        - max Rs interval - 60s
- NUD: too chatty
    - send unicast NS messages for registering IPv6 addresses
    - send messages to verify that default routers are still reachable
    - host needs to maintain "NCE" in the router
    - NUD defines probes, but can be suppressed using reachability confirmation
      (defined in 4861 and 6775). HOWEVEr does not allow for letting stuff go stale
    - registering/deregestering w/ routers using AR0 options -- why do we need these
\fi

\section{Prior Work}

\subsection{Networking Studies}
Much of the published work around RPL has focused on how RPL scales as the size
of the network increased, both in terms of path quality and convergence time to
a stable state under simulation. 

The studies contained in \cite{tripathi2010performance} draw a topology and a
link failure model from a real-world deployment and apply these measurements to
an OMNET++~\cite{omnet++} simulation that examines path quality, the overhead
of control packets and the size of routing tables. The study concluded in part
that the inclusion of a correctly configured Trickle timer \emph{has a significant influence
on the overhead of RPL control traffic.}

Some work \cite{clausen2011critical} has also been done on evaluating the
specification of RPL, using the NS2~\cite{ns2} simulator to complement
conclusions. The paper identifies the ``underspecification'' of RPL -- certain
implementation details that are not explicit in the specification document that
can significantly alter the performance of RPL.

The space and computational constraints of best-in-class hardware platforms at
the time of RPL's development made it difficult to introduce code for
evaluatating RPL on the motes themselves. Evaluations were made either through
network simulation engines or through the use of elaborate testbeds that used
back channels on deployed nodes to monitor their
behavior~\cite{fonseca2008tracking}. 

The research domain of Active Networks, first proposed in
\cite{tennenhouse2002towards}, enabled a new class of applications that utilize
computation within a network. We adopt the approach of this body of work to
form a set of tools for studying network telemetry.

The Active Node Transfer System (ANTS)~\cite{wetherall1998ants} formulated a
node runtime for an active network that enabled dynamically deploying network
protocols, such as multicast services, into a running network. Later work in
code capsules (\cite{tennenhouse1997survey}, \cite{wetherall2002active}) argued
that code capsules are best used as glue to tie together capabilities provided
by active nodes. While our work treats the routing code as fixed and does not
leverage the Active Networks notion of allowing users to customize the
forwarding of their packets, we do place active node containers in the network
and performing coordination and traffic generation through an executed capsule.
This approach is similar to Scriptroute~\cite{spring2003scriptroute}, which
enabled Internet users to conduct distributed network measurements


\section{3hop -- Zain}

\subsection{Overview -- Zain}
\if 0
- design goals
    - Control traffic should decrease as network converges
    - Support duty cycled nodes
    - Support multiple border routers
- salient features
    - Build using current IPv6 ND standard
    - Broadcast when you can
    - Use application traffic to maintain mesh
    - less control traffic overhead
\fi

\subsection{Messages -- Zain}
\if 0
- Elements
    - Mesh Prefix
        - <64 bit prefix> : <48 bit MAC> : <16 bit ID>
        - mote must discover it's prefix
    - Upward Routes
        - built thru preferred parent
    - Downward Routes
        - create neighbor table from bcasts
        - construct forwarding table of motes reachable through neighbors
    - Point to Point
        - simple triangle routing
        - sufficient enough bc max 3 hop network
    - Redundancy
        - multiple BR's
        - multiple potential parents
- Protocol
    - Router Solicitation
        - when
            - mote goes online and not part of mesh
        - includes
            - power profile
            - phase info
    - Router Advertisement (Mesh Info)
        - when
            - trickle timer
        - includes
            - mesh prefix
            - power profile
            - hop count
            - neighbor set
            - mesh schedule info
    - Router Advertisement (Announcement)
        - when
            - mote joins mesh
        - includes
            - hop count
            - preferrent parent
            - neighbor set
        - optional
\fi

\subsection{Border Router -- Zain}
\if
- why multiple borders same mesh?
    - redundancy if failures -- need multiple paths upward/downward
    - a single prefix means global source doesn't need to know which BR can reach which node
    - use application traffic, routing info to maintain state
- two main approaches
    - decentralized
        - BR reach consensus on which route is best
    - centralized
        - defer decision making/state to logically centralized coordinator
- updates
    - listen to Mesh Info/Announcement messages from mesh
\fi





\section{Evaluation Platform}

Traditionally, the storage constraints of embedded platforms have forced an opacity onto embedded networking stacks, making it difficult to add monitoring logic without removing functionality from the stack itself.
The iterative process of altering the networking stack for any sort of parametric study is intractable for more than a few parameters.
Usually, these parameter spaces are explored with the help of a network simulator such as COOJA~\cite{cooja}, OMNET++~\cite{omnet++} or NS2~\cite{ns2}.
While helpful for establishing estimates of how a protocol under a set of parameters might perform, it is difficult to evaluate the practicality of a protocol without having implemented it in a real system.
Using a modern platform with relaxed storage constraints and increased programmability, we demonstrate a novel platform for full-stack network monitoring on real-world WSN deployments.

\subsection{Hardware Platform}

The hardware platform for typically used network evaluations is the TelosB mote (used in \cite{ko2011evaluating} and modeled in COOJA for network simulations).
The TelosB, introduced in 2005~\cite{polastre2005telos},  employs a 16-bit MSP430 microprocessor with 48 KB ROM and 10 KB RAM.
This was ample space 10 years ago, but as the size and complexity of IPv6 standards and routing standards increased, the increased pressure on code space informed a set of compromises in the design of the TinyOS networking stack, our focal operating system.
These compromises and how they affect the our goal of iterative, parameter-driven evaluation are explored in more detail below.

The recently-introduced Storm platform~\cite{andersen2016system} used in this study has a 32-bit ARM Cortex M4, with 512 KB ROM and 64 KB RAM, offering much more processing power and storage capabilities.

\subsection{Software Platform}

In this section, we discuss the software platform for the Storm motes that are the nodes used in the networking study presented in Section~\ref{section:evaluation}.
The motes run a port of TinyOS 2.x, using the Berkeley Low-Power IPv6 stack (BLIP) and TinyRPL.
Over this, the nodes run the Synergy kernel with a Lua userland~\cite{andersen2016system}, which introduces a dynamic, scripting environment with visibility into the inner workings of the operating system (and thus network stack).
We extend this base functionality with:
\begin{itemize}
\item a set of new syscalls with the ability to edit routing and neighbor tables, and view control traffic and other network telemetry,
\item a userland library for conducting networking studies using an ``Active Networks'' methodology
\item a host of tricks for placing two complete routing protocols on a mote with the ability to switch without reflashing
\end{itemize}

\subsection{Synergy Kernel and Lua Userland}

\begin{table*}[ht]
\centering
\begin{tabular}{| l | l |}
\hline
\textbf{Syscall} & \textbf{Description} \\ \hline \hline
\multicolumn{2}{|c|}{\texttt{StormSysInfoP.nc}} \\ \hline
\verb`storm.net.retrystats()` & Retrieves transmission and retransmission counts \\ \hline
\verb`storm.net.thopstats()` & Retrieves number of RS and RA with 3hop options sent \\ \hline
\verb`storm.net.rplstats()` & Retrieves number of DIO, DIS and DAO messages sent \\ \hline
\verb`storm.net.stats()` & BLIP-stats: packets sent, forwarded, dropped, fragmented \\ \hline
\multicolumn{2}{|c|}{\texttt{StormRoutingTableP.nc}} \\ \hline
\verb`storm.os.addroute(pfx, len, nhop, iface)` & Add route via nexthop over given interface to routing table \\ \hline
\verb`storm.os.delroute(routekey)` & Removes given route from routing table \\ \hline
\verb`storm.os.lookuproute(pfx, pfx)` & Returns routing table entry matching the given prefix, length \\ \hline
\verb`storm.os.gettable()` & Returns all valid entries in the routing table \\ \hline
\verb`storm.os.flushroutes()` & Deletes all entries from the routing table \\ \hline
\verb`storm.os.flushneighbors()` & Deletes all entries from neighbor table \\ \hline

\end{tabular}
\caption{List of new syscalls implemented for stack monitoring. Each of the \texttt{stats()} methods has a corresponding \texttt{clear()} to erase the cumulative counts.}
\label{table:syscalls}
\end{table*}

A detailed study of a layer 3 routing protocol, such as RPL, needs to be able to pull information from layer 2 and layer 1 components in the networking stack.
The Synergy kernel exposes TinyOS functionality to userland applications through the use of syscalls, giving applications the ability to run timers, send or receive network traffic, and write to GPIO pins or other peripherals.
These syscalls are supported within TinyOS by a set of drivers, which are special TinyOS components.
We add two additional drivers: \texttt{StormRoutingTableP.nc}, which lets applications view and edit the routing and neighbor tables in BLIP; and \texttt{StormSysInfoP.nc}, which exposes routing state, packet counts, retry counts and routing control traffic counts, among other statistics.
A list of the added syscalls can be found in Table~\ref{table:syscalls}.

%(Embedded) Lua~\cite{elua} is an embeddable, lightweight, stack-based, high-level scripting language that is easily extensible in C.
%This extensibility allows us to augment

\if 0
Introduce the software stack we're working with:
-
\fi

%The storage constraints of the older platforms on which BLIP and TinyRPL were developed led to the use of compile-time configuration options (e.g. %\verb`#define` and \verb`#ifdef`) to remove unused features from the compiled stack when flashing motes.
%Although the Storm platform does not possess the same storage constraints, the status of the current codebase limits the %ability to use userland applications to configure the TinyOS networking stack.
%Refactoring the codebase to use run-time rather than compile-time parameters is a not insignificant task, but the WAX methodology could be easily extended to pass configuration options to the TinyOS networking stack to change the choice of objective function or other RPL parameters.

\if 0
talk about tinyos? blip?

Many recent features of  ipv6 nd not implemented
many constants vital to the protocol buried deep in header files with #define so that they
  can be placed into more plentiful ROM. Can't be edited at runtime
\fi

\if 0
- many prior measurements done exclusively on simulators
    - few experiments run in the real world
    - if using a Java implementation, can totally put in all features you want and
      easy to chagne parameters, but this doesn't help us evaluate how well protocol works
- Hardware:
    - advancements in hardware make


- establish the measurement apparatus for these:
    - TinyOS Kernel
    - Syscalls in Lua
        - challenges developing syscalls
    - Dynamic code loading in Lua
    - fitting everything into the mote
    - "WAX stack"
        - follow similar structure to the sensys paper
        - the components of a network experiment code
        - how to we do logging, where it goes
\fi


\section{Evaluation} \label{section:evaluation}

\bibliographystyle{IEEEtran}
\bibliography{IEEEabrv,references}

\end{document}



%%% Local Variables:
%%% mode: latex
%%% TeX-master: t
%%% End:
