\documentclass[conference]{IEEEtran}
\usepackage{enumitem}
\usepackage{graphicx}
\usepackage{url}
\usepackage{xcolor}
\usepackage{listings}
\usepackage[labelfont=bf,format=plain,font=small]{subcaption}
\usepackage[labelfont=bf,format=plain,font=small]{caption}
\usepackage{array}
\newcolumntype{L}{>{\arraybackslash}m{3cm}}
\usepackage{minted}
\newminted{sql}{mathescape, numbersep=5pt, frame=lines, framesep=2mm, fontsize=\footnotesize}
\usepackage{MnSymbol}
\def\prebreak{\raisebox{0ex}[0ex][0ex]{\ensuremath{\rhookswarrow}}}
\def\postbreak{\raisebox{0ex}[0ex][0ex]{\ensuremath{\hookrightarrow\space}}}
\def\lstbreak{\prebreak\newline\postbreak}
\lstdefinelanguage{pseudocode}{sensitive=false,morecomment=[l]{//},morestring=[b]",
                               morekeywords={if,else,while,continue,true}}
\lstset{breaklines=true,breakindent=5pt,
        postbreak=\postbreak,prebreak=\prebreak}

\newcommand{\todo}[1]{\textcolor{red}{TODO: #1}\PackageWarning{TODO:}{#1!}}


\begin{document}

\title{CS268 Project}

\author{\IEEEauthorblockN{Gabriel Fierro}
\IEEEauthorblockA{gt.fierro@berkeley.edu}
\and
\IEEEauthorblockN{Zain Amro}
\IEEEauthorblockA{zamro@berkeley.edu}
}

\maketitle

\begin{abstract}
TODO
\end{abstract}

\section{Introduction}

\section{Context and Motivation}
\if 0
- Motivation starts off by saying: look, for the very common case of indoor, low-power
  sensor networks, there is a need for routing.
- This particular case of WSN topology formation is defined by the following characteristics:
    - mains power available at some locations throughout a space: wall plugs here and there
    - It is feasible to cover the monitored space in a mesh with hops no greater than 3
    - environment can change: furniture, people -- transient connectivity
- establish the goals and parameters of our proposed routing protocol
\fi

\section{RPL}

\subsection{IPv6 Neighbor Discovery}

\section{3hop}

\section{Evaluation Platform}
\if 0
- establish the measurement apparatus for these:
    - TinyOS Kernel
    - Syscalls in Lua
        - challenges developing syscalls
    - Dynamic code loading in Lua
    - fitting everything into the mote
    - "WAX stack"
        - follow similar structure to the sensys paper
        - the components of a network experiment code
        - how to we do logging, where it goes
\fi

\section{Evaluation}

\bibliographystyle{IEEEtran}
\bibliography{IEEEabrv,references}

\end{document}



%%% Local Variables:
%%% mode: latex
%%% TeX-master: t
%%% End:
