\section{The State-of-the-Art: RPL and IPv6 ND}

Here, we present a brief overview of RFC 6550: RPL~\cite{rfc6550}, the Routing Protocol for Low Power and Lossy Networks.
RPL was developed by the IETF ROLL working group to address growing concerns of interoperability in the emerging domain of wireless sensor networks (WSNs).
The working group released several informational memos documenting motivation applications for large sensor networks in industrial~\cite{rfc5673} and urban~\cite{rfc5548} as well as for home~\cite{rfc5826} and building~\cite{rfc5867} automation.
The formulation of RPL was intended to be flexible enough in its configuration to be an effective routing solution for the entire domain of low-power and lossy networks (LLNs).

\subsection{Overview of RPL}

RPL is an IPv6-based distance-vector routing protocol designed for LLNs, a class of network in which both the routers and the interconnects are constrained.
RPL forms routes for multipoint-to-point (nodes to sink) traffic, and optionally point-to-multipoint (sink to nodes) and point-to-point (node to node) if it is configured to.
This is in contrast to CTP~\cite{ctp}, which, aside from not support IP routing, only provides support for collection (upwards) routes.
It is a stated goal of the RFC that ``RPL does not rely on any particular features of a specific link-layer technology''~\cite{rfc6550}; RPL has been implemented for power line communications (PLC) networks and for Bluetooth Low Energy (BLE).
The focus of this work is on the application of RPL to small-diameter, low-power, lossy, short-range wireless sensor networks, predicating an investigation into how well RPL utilizes the broadcast medium.

At the core of RPL is the notion of a Direction Oriented Directed Acyclic Graph (DODAG) that define routes to and from the root (sink) node.
The topology formation yields these DODAGs.
RPL supports multiple DODAGs existing in the same WSN, so each DODAG is uniquely identified by the combination of a DODAG ID --- the IPv6 address of the root node --- and a RPL Instance ID.
A RPL Instance is a collection of DODAGs that all use the same Objective Function.
Topology formation uses an Objective Function (OF) that computes a numerical Rank assigned to each node in a DODAG.
Rank strictly increases the ``farther'' a node is from the root and decreases the ``closer' it is.
During topology formation each node builds a list of candidate parents, which are nodes with ranks strictly less than its own.
The node chooses the ``best'' parent which becomes the node's default route for upwards traffic.
The maintenance of a set of candidate parents provides routing diversity that enables local rerouting in the presence of link loss or node failure without the need to reform the entire topology~\cite{hui2008ip}.

\if 0
In this section, we need to establish what RPL is, a little tiny bit of its history, and then
go into its design, how it addresses the requirements above, and how well we can predict
these might fit into our chosen scenario.

What is RPL:
  - distance-vector routing protocol designed for low-power and lossy networks, a class of
    network in which both the routers and the interconnects are constrained
  - aims to provide all three traffic patterns: up, down. point-to-point:
    - contrast to CTP which only does collection (up)
  - "RPL does not rely on any particular features of a specific link-layer technology" -RFC6550
  - means that it cannot make assumptions about those features, meaning it can lead to
    redundant design decisions
  - here we concentrate on the application of RPL to low-power, lossy, short-range wireless
    sensor networks
  - brings with it following opportunities and constraints
    - harness the broadcast medium
    - be aware of bandwidth usage, talking too much
    - specifically, "repair" mechanisms can cause a lot of traffic and end up bringing down
      other nodes because now they are overloaded

How do it work though:
- topology:
  - RPL instance = set of 1+ DODAGs sharing RPLinstance ID:
    - each operate independently, implements different objective function
  - DODAGs identify roots. uniquely identified by combination of RPL instance ID
  - rank: defines nodes individual position relative to other nodes w/ respect to the
    DODAG root. Increases farther away from root
  - DODAG root: usually the border router, configures parameters, determines when global
    repair happens
- control mesages
- how topology formation works

We can then start criticizing it:
- complexity:
  - multiple DODAGs, RPL instances, OFs
  - secure version of all headers -- not clear this could be achieved using app level
    or L2 security. To knowledge no one has implemented this
- underspecification:
  - how often to send DAO messages? relayed to the parent, constant overhead of messages
  - interoperability is a problem (contiki/tinyrpl happy together). Two implementations
    must have the same feature set, which is different than the minimal set of features
    required to have the system working:
    - storing mode vs non storing mode
- unnecessary features:
  - RPL instances -- the standard does not even talk about multiple
  - multiple objective functions: only 2 were ever implemented
  - dio/dis are very similar to ipv6 nd ra/rs messages
\fi

\subsection{Criticisms}

No matter the motivation for the constraint that RPL cannot rely on link-layer specific features, the consequence is that RPL must implement features redundant with that layer, and cannot take advantage of the nature of the medium.
In traditional routing, a ``link'' is usually thought of as connecting two hosts, but because wireless link-layers live in a broadcast medium, it no longer suffices to think of communication between hosts as independent.
This introduces a new set of tradeoffs in the design of a routing protocol: channel contention and message collision are real concerns mostly addressed by Media Access Control, but broadcast messages and hidden terminal constructions can still result in midair collisions and thus packet loss.
Additionally, because channel bandwidth is limited (250 kbits/sec for 802.15.4 radios), repair mechanisms must take care to not be too ``chatty'', lest they cause enough congestion for another node to believe it is offline and also trigger a repair.
However, the wireless medium also provides an opportunity for sharing information between nodes.

\subsection{IPv6 Neighbor Discovery -- Gabe }
\if 0
- RPL is not the only standard that would need to be revisited for wireless LLNs.
- IPv6 Neighbor Discovery is intended to provide following features:
    - router discovery: find routers on attached link
    - parameter discovery:
        - prefix discovery, link MTU, hop limit, etc
    - stateless address autoconfiguration
    - address resolution
    - neighbor unreachability detection (NUD)
    - duplicate address detection
- these are more features than are needed for WSNs because of restrictions:
    - NUD: no need to maintain connectivity to neighbors
        - use the application traffic to determine if link is still good
        - if you don't use the link, don't waste time and energy keeping it up:
            - places pressure on the channel because you can only broadcast!
        - IP is best-effort: use this leeway!
    - duplicate address detection:
        - a common way of configuring motes is SLAAC
        - one way of doing this is 64-bit prefix + 64-bit EUI-64 address
        - 48bit MAC, last 16 bits are a unique node identifier
        - upshot: can compress the address of a mote into 2 bytes (2^16-1 addresses)
        - guaranteed unique address: no need to check for dups
- constants: too chatty?
    - rfc6775 attempts to adapt for 6LowPANs:
    - "Minimize signaling by avoiding the use of multicast
      flooding and reducing the use of link-scope multicast messages"
    - original constants: cite them
    - rfc6775 new constants:
        - max RAs: 3
        - min delay btwn RA: 10s
        - max RA delay time 2 sec
        - RS interval: 10s
        - max RS : 3
        - max Rs interval - 60s
- NUD: too chatty
    - send unicast NS messages for registering IPv6 addresses
    - send messages to verify that default routers are still reachable
    - host needs to maintain "NCE" in the router
    - NUD defines probes, but can be suppressed using reachability confirmation
      (defined in 4861 and 6775). HOWEVEr does not allow for letting stuff go stale
    - registering/deregestering w/ routers using AR0 options -- why do we need these
\fi
