\section{Prior Work}

\subsection{Networking Studies}
Much of the published work around RPL has focused on how RPL scales as the size
of the network increased, both in terms of path quality and convergence time to
a stable state under simulation. 

The studies contained in \cite{tripathi2010performance} draw a topology and a
link failure model from a real-world deployment and apply these measurements to
an OMNET++~\cite{omnet++} simulation that examines path quality, the overhead
of control packets and the size of routing tables. The study concluded in part
that the inclusion of a correctly configured Trickle timer \emph{has a significant influence
on the overhead of RPL control traffic.}

Some work \cite{clausen2011critical} has also been done on evaluating the
specification of RPL, using the NS2~\cite{ns2} simulator to complement
conclusions. The paper identifies the ``underspecification'' of RPL -- certain
implementation details that are not explicit in the specification document that
can significantly alter the performance of RPL.

The space and computational constraints of best-in-class hardware platforms at
the time of RPL's development made it difficult to introduce code for
evaluatating RPL on the motes themselves. Evaluations were made either through
network simulation engines or through the use of elaborate testbeds that used
back channels on deployed nodes to monitor their
behavior~\cite{fonseca2008tracking}. 

The research domain of Active Networks, first proposed in
\cite{tennenhouse2002towards}, enabled a new class of applications that utilize
computation within a network. We adopt the approach of this body of work to
form a set of tools for studying network telemetry.

The Active Node Transfer System (ANTS)~\cite{wetherall1998ants} formulated a
node runtime for an active network that enabled dynamically deploying network
protocols, such as multicast services, into a running network. Later work in
code capsules (\cite{tennenhouse1997survey}, \cite{wetherall2002active}) argued
that code capsules are best used as glue to tie together capabilities provided
by active nodes. While our work treats the routing code as fixed and does not
leverage the Active Networks notion of allowing users to customize the
forwarding of their packets, we do place active node containers in the network
and performing coordination and traffic generation through an executed capsule.
This approach is similar to Scriptroute~\cite{spring2003scriptroute}, which
enabled Internet users to conduct distributed network measurements
