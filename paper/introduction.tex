\section{Introduction}

It has been several years since RPL was introduced as a routing protocol for the kinds of links used in wireless sensor networks (WSNs).
While various commercial products have introduced some form of IPv6 over LoWPAN, these operate in stand-alone settings, such as automated meter reading networks, and there is essentially nothing in the literature regarding specifics or performance of their routing protocols.
Open implementations of RPL are available, but there is a pronounced lack of study of their empirical behavior, particularly in the common case of small (10s of nodes) shallow (one to three hop) networks.

The original vision of this project was to develop a new routing protocol --- \texttt{3hop} --- for this common but understated scenario and to augment a discussion of the design and implementation of this new protocol with a comparative study of RPL.


\if 0
- Motivation starts off by saying: look, for the very common case of indoor, low-power
  sensor networks, there is a need for routing.
- what do WSN nodes look like?
    - http://www.airccse.org/journal/ijcses/papers/1110ijcses06.pdf
    - low bandwidth, low power, low memory, short range (10-50m)
    - unreliable -- things crash, run out of power -> topology changes!
      - need to adapt to failures
    - self-configuring: nodes will need to discover the topology.
    - channel utilization:
      - want to make efficient use of bandwidth by reducing control traffic
        necessary for maintaining the mesh
      - has upshot of saving power too
      - this is also addressed at MAC layer, which we do not address
- This particular case of WSN topology formation is defined by the following characteristics:
    - SMALL: order 10s of nodes, over a small geographic area: a building or collection of rooms
    - mains power available at some locations throughout a space: wall plugs here and there
    - It is feasible to cover the monitored space in a mesh with hops no greater than 3
    - environment can change: furniture, people -- transient connectivity
- establish the goals and parameters of our proposed routing protocol
- Firstly, what does a routing protocol have to do?
    - mote must discover the prefix
      - simplifying assumptions: slaac, using 48-bit MAC and 16 bit node id
    - build upward routes: multipoint-to-point
      - list some ways to do this?
    - build downward routes
      - need for acknowledgements, actuations
    - point to point routes
      - rarer, but in small networks w/ actuators and sensors, this may become more common.
    - prepare for failure:
      - provide a means of repairing the network
    - useful parts: border router, providing physical translation in/out of
      the network. Handles prefix compression/expansion for 6lowpan, generally
      more powerful, powered. Can have large bandwidth backbone
\fi

