\section{Evaluation Platform}

Traditionally, the storage constraints of embedded platforms have forced an opacity onto embedded networking stacks, making it difficult to add monitoring logic without removing functionality from the stack itself.
The iterative process of altering the networking stack for any sort of parametric study is intractable for more than a few parameters.
Usually, these parameter spaces are explored with the help of a network simulator such as COOJA~\cite{cooja}, OMNET++~\cite{omnet++} or NS2~\cite{ns2}.
While helpful for establishing estimates of how a protocol under a set of parameters might perform, it is difficult to evaluate the practicality of a protocol without having implemented it in a real system.
Using a modern platform with relaxed storage constraints and increased programmability, we demonstrate a novel platform for full-stack network monitoring on real-world WSN deployments.

\subsection{Hardware Platform}

The hardware platform for typically used network evaluations is the TelosB mote (used in \cite{ko2011evaluating} and modeled in COOJA for network simulations).
The TelosB, introduced in 2005~\cite{polastre2005telos},  employs a 16-bit MSP430 microprocessor with 48 KB ROM and 10 KB RAM.
This was ample space 10 years ago, but as the size and complexity of IPv6 standards and routing standards increased, the increased pressure on code space informed a set of compromises in the design of the TinyOS networking stack, our focal operating system.
These compromises and how they affect the our goal of iterative, parameter-driven evaluation are explored in more detail below.

The recently-introduced Storm platform~\cite{andersen2016system} used in this study has a 32-bit ARM Cortex M4, with 512 KB ROM and 64 KB RAM, offering much more processing power and storage capabilities.

\subsection{Software Platform}
\if 0
talk about tinyos? blip?

Many recent features of  ipv6 nd not implemented
many constants vital to the protocol buried deep in header files with #define so that they
  can be placed into more plentiful ROM. Can't be edited at runtime
\fi

\if 0
- many prior measurements done exclusively on simulators
    - few experiments run in the real world
    - if using a Java implementation, can totally put in all features you want and
      easy to chagne parameters, but this doesn't help us evaluate how well protocol works
- Hardware:
    - advancements in hardware make


- establish the measurement apparatus for these:
    - TinyOS Kernel
    - Syscalls in Lua
        - challenges developing syscalls
    - Dynamic code loading in Lua
    - fitting everything into the mote
    - "WAX stack"
        - follow similar structure to the sensys paper
        - the components of a network experiment code
        - how to we do logging, where it goes
\fi
